I risultati ottenuti dall'esperimento sono stati abbastanza in linea con quelli che ci si aspettava.\\

Per quanto riguarda la calibrazione in tempo ed in energia, tutti i risultati sono stati perfettamente in linea con le aspettative (per esempio le due calibrazioni in energia
dei rivelatori sono risultate simili), e la calibrazione in tempo coincide con le aspettative. In tutti i grafici in energia è stato abbastanza semplice riconoscere le spalle
Compton e non si sono riconosciuti fenomeni particolari.\\

Per quanto riguarda lo studio della risoluzione utilizzando la sorgente di Cobalto al posto di quella di sodio si è notato effettivamente che lo spettro coincideva con
quello teorico; la risoluzione tende a migliorare all'aumentare dell'energia come ci si aspetta (infatti più energetici sono i fotoni più alta è la probabilità di interazione
all'interno del rivelatore, e se la probabilità di interazione è più alta i due punti di interazione all'interno dei due rivelatori saranno più simili, il che porta
ad una risoluzione minore).\\

A proposito della misura della velocità della luce, effettivamente il valore ottenuto è abbastanza vicino al valore presente in letteratura, anche se l'errore è stato
evidentemente sottostimato, probabilmente a causa di una sottostima di tutti gli effetti strumentali legati alla strumentazione elettronica: la differenza temporale
tra i due segnali è effettivamente molto piccola per poter associare una misura molto precisa ad essa, e l'errore associato è di ordini di grandezza inferiore a quello che
ci si aspetta, risulta poco plausibile pensare che con un simile apparato strumentale si possa avere un errore su una misura temporale dell'ordine dei picosecondi.\\

Lo studio del sistema di acquisizione digitale ha rivelato come il digitizer utilizzato non sia adeguato alle misurazioni prese: infatti sebbene i grafici rivelino una
forma più \textit{appuntita} (e quindi sembrano associare errori più piccoli), non sono effettivamente buone gaussiane. Probabilmente tale problema è da imputare al
campionamento non sufficiente dal digitizer stesso, come evidenziato nella sezione apposita, a seconda del punto di campionamento lo \textit{zero crossing} avviene in
punti diversi, e tale errore è di stima molto difficile. Questo problema rende meno importante il confronto tra le risoluzioni nei due metodi: sebbene la FWHM sia minore
nel caso dell'analisi digitale, essa presenta fenomeni strumentali non imputabili a effetti fisici, e quindi non permette di prendere dati attendibili, perciò si preferisce
lo studio analogico.
