L'apparato strumentale consiste di una serie di moduli (amplificatori di alto voltaggio, un fan in/out, un amplificatore analogico, una cassetta di ritardi, un CFTD e un TAC),
due scintillatori organici di 5 cm ciascuno collegato a un fotomoltiplicatore XP2020,  un oscilloscopio, un ADC ed un digitizer.
Durante la prima sessione di laboratorio si è preso confidenza con tutte le parti dell'apparato
strumentale e si è fatta la calibrazione di tale apparato. Come prima cosa si sono collegati i rivelatori al fan in/out e il segnale è stato mandato all'oscilloscopio,
in modo che fosse possibile vedere il segnale e misurarne ampiezza, tempo di salita e tempo di discesa e rumore. Poi si è ripetuto il procedimento collegando il rivelatore, e si
sono anche collegati il CFD e il TAC per la prova della misura di tempo. Una volta capito il funzionamento dei vari moduli, si è proceduto con la regolazione della soglia:
dopo aver collegato tutti i moduli all'oscilloscopio si è visto il segnale del CFTD in concomitanza con il segnale in uscita direttamente dall'amplificatore: in
questo modo è stato possibile verificare che non ci fossero falsi eventi, e regolando un trimmer si è regolata la soglia del CFTD in modo che fosse la più bassa possibile
ma senza registrare falsi eventi a causa del rumore elettronico. Infine si è preso un file di prova per verificare il corretto funzionamento del sistema di misura.\\

Subito dopo si è passati alla vera e propria calibrazione in energia: per prendere i dati si è impostato il sistema di acquisizione in modo che triggherasse singolarmente
sui segnali provenienti dal primo rivelatore e poi su quelli provenienti dal secondo rivelatore. Si sono quindi acquisiti due campioni di dati per una durata di circa 
15 minuti l'uno, che
contenessero entrambe le spalle Compton del decadimento del sodio, da usare per la calibrazione in energia.\\

Durante la seconda sessisone si è passati alla calibrazione in tempo e alla misura del ritardo de cavi forniti. Per farla, si sono prese misure analoghe a quelle prese per la calibrazione in
energia andando a modificare i ritardi introdotti, e poi si sono studiati i grafici risultanti. Dopodichè si sono misurati i ritardi legati ai cavetti aggiungendoli
in serie all'uscita della scatola dei ritardi, e si sono presi altri campioni.\\

Successivamente si è passati allo studio della risoluzione temporale in funzione del delay: si è cambiato il cavo del delay utilizzato dal CFTD e si sono studiate le
risoluzioni al variare della lunghezza di tale cavo, in modo da trovare la lunghezza ottimale per la presa dati successiva. Inoltre, collegando una delle uscite del
CFTD assieme al CF MONITOR all'oscilloscopio si è modificato il trimmer WALK ADJ in modo tale che l'intersezione dei segnali bipolari visti sull'oscilloscopio
coincidesse con la baseline del segnale stesso, e che quindi la misura di tempo fosse la migliore possibile.\\

Una volta effettuate tutte le calibrazioni e le regolazioni necessarie si è deciso di prendere una misura che potesse dare una stima della risoluzione temporale al
variare dell'energia, così si è sostituita la sorgente di $^{22}\text{Na}$ con una sorgente di $^{60}\text{Co}$ (che emette fotoni in coincidenza più energetici, sebbene non collineari)
e si è verificato che tutte le regolazioni dell'apparato fossero quelle ottimali per prendere una misura che fosse la migliore possibile. Tale misura è stata presa
nell'intervallo tra la seconda e la terza sessione di laboratorio (è servito un campione più  lungo considerando che i fotoni erano collineari solo come accidente).\\

La terza sessione è stata dedicata alla misura della velocità della luce e all'ulitizzo del sistema di acquisizione digitale. Per misurare la velocità della luce,
come prima cosa, si è posizionata la sorgente di $^{22}\text{Na}$, che decade attraverso due fotoni collineari e si è controllato che tutti i parametri della strumentazione
(in particolare i ritardi inseriti nella cassetta, il filo del delay del CFTD, la soglia e il Walk Adj) fossero quelli ottimali per la presa dati. Dopodiché si sono allontanati
i due rivelatori il più possibile e
si è avvicinata la sorgente al rivelatore 1, triggherando su tale rivelatore si sono osservati i segnali in coincidenza; poi si sono cambiati i valori dell'amplificatore
e della cassetta dei ritardi per fare in modo che i segnali avessero una larghezza di circa 150 ns e fossero distanziati il più possibile. Settato l'apparato, si è presa una
misura di circa un'ora con la sorgente vicina al primo rivelatore e un'altra con la sorgente vicina al secondo rivelatore, e si sono misurate le distanze caratteristiche del 
sistema.\\

Come ultima cosa si è passati al sistema di acquisizione digitale: si sono collegati i cavi del sistema (preimpostato) di acquisizione digitale
e si è preso un campione in uscita dai due rivelatori indipendentemente lungo circa una ventina di minuti.
